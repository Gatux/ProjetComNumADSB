\documentclass[11pt]{article}
\usepackage[left=1in, right=1in, top=1in, bottom=1in]{geometry}
\usepackage{layout}
\usepackage{ucs}
\usepackage[french]{babel}
\usepackage[latin1]{inputenc}
\usepackage[T1]{fontenc}
\usepackage{titlesec}
\usepackage{graphicx}
\usepackage{amssymb}
\usepackage{amsmath}
\usepackage{dsfont}
\usepackage{caption}
\usepackage{subcaption}
\usepackage{array}
\usepackage{stmaryrd}
\usepackage{fourier}
\usepackage[T1]{fontenc}
\usepackage{enumitem}
\usepackage[frenchb]{babel}
\usepackage{pgfplots}



\title{\textbf{Compte rendu de projet - TS226}\\Simulation d'un �metteur / r�cepteur ADS-B}
\author{Maxime PETERLIN - \texttt{maxime.peterlin@enseirb-matmeca.fr}\\
Gabriel VERMEULEN - \texttt{gabriel@vermeulen.email} \\\\{ENSEIRB-MATMECA, Bordeaux}}
\date{19 janvier 2014}


\begin{document}

\maketitle
\tableofcontents

\newpage

\section*{Introduction}
\addcontentsline{toc}{section}{Introduction}

\section{�tude th�orique}

	La modulation en position d'amplitude est utilis�e pour la transmission de signaux ADS-B. 
	% Figures ppm
	On a alors l'enveloppe complexe du signal �mis qui est la suivante :
	\[
		s_l(t) = \sum \limits_{k \in \mathbb{Z}} p_{b_k}(t-kT_s)
	\]
	avec $T_s = 1\mu s$ le temps de l'impulsion �l�mentaire et
	\[
		p_{b_k}(t) =
		\begin{cases}
				p_0(t)\text{, si } b_k = 0\\
				p_1(t)\text{, si } b_k = 1
		\end{cases}
	\]
	\newline
	
	$s_l$ peut �galement s'exprimer en fonction des symboles �mis $A_k$ et de la forme d'onde biphase donn�e ci-dessous.
	% Figure de la forme d'onde biphase
	\begin{align}
		s_l(t) &= \sum \limits_{k \in \mathbb{Z}} p_{b_k}(t_kT_s) \\
		&= \sum \limits_{\substack{k \in \mathbb{Z}\\b_k = 0}} p_0(t-kT_s) + \sum \limits_{\substack{k \in \mathbb{Z}\\b_k = 1}} p_1(t-kT_s)\\
		&= 0.5 + \sum \limits_{b_k = 0} p(t-kT_s) - \sum \limits_{b_k = 1} p(t-kT_s)\\
		s_l(t) &= 0.5 + \sum \limits_{b_k = 0} A_k p(t-kT_s) + \sum \limits_{b_k = 1} A_k p(t-kT_s)
	\end{align}
Finalement, on obtient :
	\[
	\boxed{s_l(t) = \0.5 + \sum \limits_{k \in \mathbb{Z}} A_k p(t-kT_s) }	
	\]
avec 
	\[
		A_k =
		\begin{cases}
				
				1\text{, si } b_k = 0\\
				-1\text{, si } b_k = 1
		\end{cases}
	\]
	\newline
	
	En r�ception, on aura les 
	
	

\section{�tude algorithmique}

\section{Impl�mentation sous \textsc{MATLAB}}

\section{R�sultats}

\section*{Conclusion}
\addcontentsline{toc}{section}{Conclusion}
	

\end{document}
